\documentclass[a4paper,oneside,onecolumn,12pt]{LegrandOrangeBook}
%\documentclass[a4paper,12pt]{article}
\usepackage[utf8]{inputenc}
\usepackage[T1]{fontenc}
\def\magyarOptions{chapterhead=unchanged}
%\usepackage[magyar]{babel}
\usepackage{longtable}
%\usepackage{amsthm}
% \usepackage{amsmath}
% \usepackage{amsfonts}
% \usepackage{txfonts}
% \usepackage{pxfonts}
% %\usepackage{eufrak}
% \usepackage{mathptmx}
\usepackage{setspace}
% \usepackage{graphicx}
\usepackage{parskip} %paragraph között legyen spacing
% \usepackage[export]{adjustbox}%kép pozicionálás
\usepackage{minted}
\usepackage{svg}
% \usepackage{hyperref}
% \usepackage{multimedia}
% \usepackage{wrapfig}
% \usepackage{csquotes}
\usepackage{ragged2e}
\usepackage{float} % For anchoring table locations

% electronic ISBN
%\usepackage[SC5b,ISBN=000-00-000-0000-0]{ean13isbn}

% printed ISBN
\usepackage[SC5b,ISBN=000-00-000-0000-0]{ean13isbn}

\newcommand{\blankpage}{%
    \newpage
    \thispagestyle{empty}
    \mbox{}
    \newpage
}

\newcommand{\comment}[1]{{\textcolor{red}{#1}}}
\newcommand{\commentaron}[1]{{\textcolor{blue}{#1}}}

% Book information for PDF metadata, remove/comment this block if not required 
\hypersetup{
	pdftitle={Real-time stock market price data analysis using neural networks}, % Title field
	pdfauthor={Bc. Eugen Fekete}, % Author field
	pdfsubject={Diplomamunka}, % Subject field
	pdfkeywords={key1, key2, key3, key4}, % Keywords
	pdfcreator={LaTeX}, % Content creator field
}

\definecolor{ocre}{RGB}{243, 102, 25} % Define the color used for highlighting throughout the book

\chapterspaceabove{6.5cm} % Default whitespace from the top of the page to the chapter title on chapter pages
\chapterspacebelow{6.75cm} % Default amount of vertical whitespace from the top margin to the start of the text on chapter pages

\onehalfspacing
%\doublespacing
\frenchspacing

\hypersetup{
    colorlinks=true,
    linkcolor=blue,
    filecolor=magenta,      
    urlcolor=cyan,
    pdftitle={Real-time stock market price data analysis using neural networks} }
\urlstyle{same}

%% \usepackage[backend=biber, style=numeric-comp, sorting=none,]{biblatex}
%% \PassOptionsToPackage{sorting=none,language=english}{biblatex}
\ExecuteBibliographyOptions{sorting=nyt}
\addbibresource{manuscript.bib}

\begin{document}

%%% Borító
\thispagestyle{empty}
\begin{minipage}[c][\textheight][c]{\textwidth}
	{\centering
	\includegraphics[keepaspectratio,width=3cm]{SelyeBanner.png}\\
	\vskip0.5cm
	{\LARGE UNIVERZITA J. SELYEHO}\\
	\vskip0.5cm
	{\LARGE SELYE JÁNOS EGYETEM}\\
    \vskip0.5cm
	{\large Fakulta ekonómie a informatiky}\\
	\vskip0.5cm
	{\large Gazdaságtudományi és Informatikai Kar}\\
	\vfill
	{\Huge Real-time stock market price data analysis using neural networks}\\
	%\vskip2cm
	%{\Huge A ZÁRÓDOLGOZAT CÍME}\\
	%\vfill
	Diplomamunka\\
	Bc. Eugen Fekete \\
    \ISBN\\
	\hfill\the\year{}, Komárno\hfill
	}
\end{minipage}

\cleardoublepage
\begingroup
\makeatletter
\let\ps@plain\ps@empty
\begin{minipage}[c][\textheight][c]{\textwidth}
	{\centering
	{\large UNIVERZITA J. SELYEHO\\SELYE JÁNOS EGYETEM}\\
	\vskip0.5cm
	{\ NÁZOV FAKULTY\\Fakulta ekonómie a informatiky\\Gazdaságtudományi és Informatikai Kar}\\
	\vfill
	{\Large NÁZOV PRÁCE\\Analýza údajov o cenách na burze v reálnom čase pomocou neurónových sietí }\\
	\vfill
	\thispagestyle{empty}
	\begin{tabular}{ll}
		Študijný program:    & Aplikovaná informatika \\
		Tanulmányi program:  & Alkalmazott Informatika\\
		Študijný odbor:      & Informatika\\
		Tanulmányi szak:     & Informatika\\
		Školiteľ:            & László Marák, PhD.\\
		Témavezető:          & László Marák, PhD.\\
		Školiace pracovisko: & Katedra informatiky\\
		Tanszék megnevezése: & Informatikai Tanszék\\
	\end{tabular}
	\vfill
	Označenie typu práce - Diplomamunka\\
	Bc. Eugen Fekete\\
    \ISBN \\
	\hfill\the\year{}, Komárno\hfill
	}
\end{minipage}
\endgroup
{
\hspace*{-2cm}
%\includegraphics[keepaspectratio, width=17cm]{./zadanie-zp_17422.pdf}
Ide jön az aláírt témakiírás
\pagebreak
}
\tableofcontents %% opcionális
\pagebreak
\listoffigures  %% opcionális
\addcontentsline{toc}{section}{Ábrák jegyzéke}

\pagebreak

\newcommand{\chpt}[1]{\chapter*{#1}\addcontentsline{toc}{section}{#1}}
% \chapterimage{kep/header.png} % Chapter heading image

\chpt{Feladatkiírás}
% \addcontentsline{toc}{chapter}{Feladatkiírás}

A szerző egy feladatot oldott meg. Még egy kicsit hosszabb, még egy kicsit hosszabb, még egy kicsit hosszabb, még egy kicsit hosszabb, még egy kicsit hosszabb, még egy kicsit hosszabb, még egy kicsit hosszabb, még egy kicsit hosszabb, még egy kicsit hosszabb, még egy kicsit hosszabb, még egy kicsit hosszabb, még egy kicsit hosszabb, még egy kicsit hosszabb, még egy kicsit hosszabb, még egy kicsit hosszabb, még egy kicsit hosszabb, még egy kicsit hosszabb, még egy kicsit hosszabb, még egy kicsit hosszabb, még egy kicsit hosszabb, még egy kicsit hosszabb, még egy kicsit hosszabb, még egy kicsit hosszabb, még egy kicsit hosszabb, még egy kicsit hosszabb, még egy kicsit hosszabb, még egy kicsit hosszabb, még egy kicsit hosszabb, még egy kicsit hosszabb, még egy kicsit hosszabb, még egy kicsit hosszabb, még egy kicsit hosszabb, még egy kicsit hosszabb, még egy kicsit hosszabb, még egy kicsit hosszabb, még egy kicsit hosszabb, még egy kicsit hosszabb, még egy kicsit hosszabb, még egy kicsit hosszabb, még egy kicsit hosszabb, még egy kicsit hosszabb, még egy kicsit hosszabb, még egy kicsit hosszabb, még egy kicsit hosszabb, még egy kicsit hosszabb, még egy kicsit hosszabb, még egy kicsit hosszabb,  

% \chapterimage{kep/header2.png} % Chapter heading image
\chpt{Opis práce}
% \addcontentsline{toc}{chapter}{Opis práce}

Autor vyrešil úlohu, ešte trošku dlhšie, ešte trošku dlhšie, ešte trošku dlhšie, ešte trošku dlhšie, ešte trošku dlhšie, ešte trošku dlhšie, ešte trošku dlhšie, ešte trošku dlhšie, ešte trošku dlhšie, ešte trošku dlhšie, ešte trošku dlhšie, ešte trošku dlhšie, ešte trošku dlhšie, ešte trošku dlhšie, ešte trošku dlhšie, ešte trošku dlhšie, ešte trošku dlhšie, ešte trošku dlhšie, ešte trošku dlhšie, ešte trošku dlhšie, ešte trošku dlhšie, ešte trošku dlhšie, ešte trošku dlhšie, ešte trošku dlhšie, ešte trošku dlhšie, ešte trošku dlhšie, ešte trošku dlhšie, ešte trošku dlhšie, ešte trošku dlhšie, ešte trošku dlhšie, ešte trošku dlhšie, ešte trošku dlhšie, ešte trošku dlhšie, ešte trošku dlhšie, ešte trošku dlhšie, ešte trošku dlhšie, ešte trošku dlhšie, ešte trošku dlhšie, ešte trošku dlhšie, ešte trošku dlhšie, ešte trošku dlhšie, ešte trošku dlhšie, ešte trošku dlhšie, ešte trošku dlhšie
\pagebreak

%\chapterimage{kep/header3.png} % Chapter heading image
\chpt{Abstrakt}\label{sec:abstrakt}
% \addcontentsline{toc}{chapter}{Abstrakt}

ešte trošku dlhšie, ešte trošku dlhšie, ešte trošku dlhšie, ešte trošku dlhšie, ešte trošku dlhšie, ešte trošku dlhšie, ešte trošku dlhšie, ešte trošku dlhšie, ešte trošku dlhšie, ešte trošku dlhšie, ešte trošku dlhšie, ešte trošku dlhšie, ešte trošku dlhšie, ešte trošku dlhšie, ešte trošku dlhšie, ešte trošku dlhšie, ešte trošku dlhšie, ešte trošku dlhšie, ešte trošku dlhšie, ešte trošku dlhšie, ešte trošku dlhšie, ešte trošku dlhšie, ešte trošku dlhšie, ešte trošku dlhšie, ešte trošku dlhšie, ešte trošku dlhšie, ešte trošku dlhšie, ešte trošku dlhšie, ešte trošku dlhšie, ešte trošku dlhšie, ešte trošku dlhšie, ešte trošku dlhšie, ešte trošku dlhšie, ešte trošku dlhšie, ešte trošku dlhšie, ešte trošku dlhšie, ešte trošku dlhšie, ešte trošku dlhšie, ešte trošku dlhšie, ešte trošku dlhšie, ešte trošku dlhšie, ešte trošku dlhšie, ešte trošku dlhšie, ešte trošku dlhšie

\textbf{Kľúčové slová: klúč1, klúč2, klúč3, }
\pagebreak

%\chapterimage{kep/header4.png} % Chapter heading image
\chpt{Absztrakt}\label{sec:absztrakt}
% \addcontentsline{toc}{chapter}{Absztrakt}

A szerző egy feladatot oldott meg. Még egy kicsit hosszabb, még egy kicsit hosszabb, még egy kicsit hosszabb, még egy kicsit hosszabb, még egy kicsit hosszabb, még egy kicsit hosszabb, még egy kicsit hosszabb, még egy kicsit hosszabb, még egy kicsit hosszabb, még egy kicsit hosszabb, még egy kicsit hosszabb, még egy kicsit hosszabb, még egy kicsit hosszabb, még egy kicsit hosszabb, még egy kicsit hosszabb, még egy kicsit hosszabb, még egy kicsit hosszabb, még egy kicsit hosszabb, még egy kicsit hosszabb, még egy kicsit hosszabb, még egy kicsit hosszabb, még egy kicsit hosszabb, még egy kicsit hosszabb, még egy kicsit hosszabb, még egy kicsit hosszabb, még egy kicsit hosszabb, még egy kicsit hosszabb, még egy kicsit hosszabb, még egy kicsit hosszabb, még egy kicsit hosszabb, még egy kicsit hosszabb, még egy kicsit hosszabb, még egy kicsit hosszabb, még egy kicsit hosszabb, még egy kicsit hosszabb, még egy kicsit hosszabb, még egy kicsit hosszabb, még egy kicsit hosszabb, még egy kicsit hosszabb, még egy kicsit hosszabb, még egy kicsit hosszabb, még egy kicsit hosszabb, még egy kicsit hosszabb, még egy kicsit hosszabb, még egy kicsit hosszabb, még egy kicsit hosszabb, még egy kicsit hosszabb, 

\textbf{Kulcsszavak: kulcs1, kulcs2, kulcs3}

\pagebreak

%\chapterimage{kep/header.png} % Chapter heading image
\chpt{Abstract}
Little longer, Little longer, Little longer, Little longer, Little longer, Little longer, Little longer, Little longer, Little longer, Little longer, Little longer, Little longer, Little longer, Little longer, Little longer, Little longer, Little longer, Little longer, Little longer, Little longer, Little longer, Little longer, Little longer, Little longer, Little longer, Little longer, Little longer, Little longer, Little longer, Little longer, Little longer, Little longer, Little longer, Little longer, Little longer, Little longer, Little longer, Little longer, Little longer, Little longer, Little longer, Little longer, Little longer, Little longer, Little longer, Little longer, Little longer, Little longer, Little longer, Little longer, Little longer, Little longer, Little longer, Little longer, Little longer, Little longer, Little longer, 

\textbf{Keywords: key1, key2, key3}

\pagebreak

\chapter*{Introduction}
\addcontentsline{toc}{chapter}{Introduction}
\markboth{}{\sffamily\normalsize{Introduction}}
% \begin{eBox}
%     Fontos megjegyzés, vagy állítás
% \end{eBox}
% \commentaron{szürke háttérrel kicsit fura nekem}\comment{válassz tetszőleges hátteret}

% \begin{figure}
%   \begin{tBox}
%     \centering{\LARGE{\url{https://google.com/}}}
%   \end{tBox}
%   \caption{Felirat}\label{fig:link}
% \end{figure}  

Machine learning (ML) plays a pivotal role in many areas of modern sciences, whether in industry, healthcare, finance and other fields. It can be used to provide a better service for the users of a search engine, a social media site or a media service provider by learning from the behaviour of the average user, predict stock prices within a specific time interval based on company performance measures and economic data, identify the risk factors for certain health conditions derived from clinical and demographic variables, identify the characters in a handwritten address from a digitized image, and so on. \cite{TESL}\\
The main objective of ML is to find rules or patterns in data to achieve certain goals. In the financial world, for example, this might involve extracting useful information from the available data to support or automate investment activities. These activities include observing the market and placing buy or sell orders based on the conclusions drawn. \cite{MLAT}

% \chapterimage{kep/header2.png} % Chapter heading image
\chapter{Elméleti rész}
Lorem ipsum dolores

	\section{Machine Learning}
	The more common way of making a computer do work is to execute a computer program created by a human programmer. This program contains the steps and rules that turn input data into the appropriate answers, called output data. Machine learning mixes up these steps: the machine examines the input and output data, and tries to figure out what the rules should be. A system working like this is said to be trained rather than programmed. It is during the training process that the system identifies these rules by learning the patterns and relationships in the available data. \cite{DLP}\\
	Learning can be described using the definition provided by the renowned computer scientist and machine learning researcher, Tom Michael Mitchell:
	\begin{quote}
		"A computer program is said to learn from experience E with respect to some class of tasks T and performance measure P, if its performance at tasks in T, as measured by P, improves with experience E." (Tom M. Mitchell, 1997, p. 2)\cite{ML}
	\end{quote}
	As an example, a text recognition program, a so called Optical Character Recognition (OCR) software can be presented. The main goal of such a program is to correctly recognize and convert handwritten characters into digitized text. A collection of texts written in various styles is presented to the OCR software. This collection, which the system uses to learn, is called the training set, where each instance is labeled appropriately. The actual machine learning part of the software that learns and makes predictions is called the model. In this example, the task T is to recognize handwritten characters and correctly classify them, the experience E is the training set provided for learning and the performance measure P could be the accuracy of the recognition.\\
	The example mentioned described a ML system performing supervised learning and solving a classification problem. We talk about supervised learning when a training set with appropriately labeled data is available for the learning process. Two other well known types of learning are unsupervised learning, where no training set with labeled data is available, and reinforcement learning, where a software agent learns rules by interacting with its environment. A classification problem is a problem where each input can be sorted into discrete number of classes. In the previous example each letter in a text can be classified as one of the letters of the alphabet. In contrast, when predicting land prices, we do not expect discrete labels as outputs, so we can't speak of classification problems. This is known as a regression problem and we expect continuous numerical values as outputs, for which a regression algorithm is used.

		\subsection{The learning process}
		Machine learning systems can be grouped based on the type and amount of supervision received during the training. The most common types of learning are:
		\begin{itemize}[noitemsep]
			\item Supervised learning
			\item Unsupervised learning
			\item Reinforcement learning
		\end{itemize}
		\cite{HMLSKT}
			
			\subsubsection{Supervised learning}
			For supervised learning a training set is available. The training set consists of numerous observations with predefined inputs and a corresponding output. The inputs are called predictors (also called features) and the output is referred to as the response. The observations represents individual data instances. \cite{TESL}\\
			As an example lets say we want to predict house prices and we have the following training set:\\
			\begin{table}[H]
			\begin{center}
			\begin{tabular}{|c|c|c||c|}
				\hline
				\textbf{Size ($m^2$)} & \textbf{Number of rooms} & \textbf{Location} & \textbf{Price (€)}\\
				\hline
				250 & 4 & Bratislava & 412000\\
				\hline
				390 & 6 & Komárno & 186000\\
				\hline
				180 & 4 & Košice & 375000\\
				\hline
				\multicolumn{4}{|c|}{\cdots} \\
				\hline
			\end{tabular}
			\end{center}
			\caption{Example training set with arbitrary values}
			\end{table}
			In Table 1.1 each row corresponds to a single house for sale. These houses are the observations and each column (excluding the last column) of the table, which represents the attributes of the houses, is a predictor. The last column is the output or the response.\\
			

\section{Mégegy alfejezet}

felsorolás

\begin{itemize}
    \item LLE - {\it low-level} emuláció - alacsony szintű emulátorok %- ciklus pontosság
    \item HLE - {\it high-level} emuláció - magas szintű emulátorok %- absztrakciók
\end{itemize}


\subsection{Al-alfejezet}

Még egy kicsit hosszabb, még egy kicsit hosszabb, még egy kicsit hosszabb, még egy kicsit hosszabb, még egy kicsit hosszabb, még egy kicsit hosszabb, még egy kicsit hosszabb, még egy kicsit hosszabb, még egy kicsit hosszabb, még egy kicsit hosszabb, még egy kicsit hosszabb, még egy kicsit hosszabb, még egy kicsit hosszabb, még egy kicsit hosszabb, még egy kicsit hosszabb, még egy kicsit hosszabb, még egy kicsit hosszabb, még egy kicsit hosszabb, még egy kicsit hosszabb, még egy kicsit hosszabb, még egy kicsit hosszabb, még egy kicsit hosszabb, még egy kicsit hosszabb, még egy kicsit hosszabb, még egy kicsit hosszabb, még egy kicsit hosszabb, még egy kicsit hosszabb\footnote{lábjegyzet}, még egy kicsit hosszabb, még egy kicsit hosszabb, még egy kicsit hosszabb, még egy kicsit hosszabb, még egy kicsit hosszabb, még egy kicsit hosszabb, még egy kicsit hosszabb, még egy kicsit hosszabb, még egy kicsit hosszabb, még egy kicsit hosszabb, még egy kicsit hosszabb, még egy kicsit hosszabb, még egy kicsit hosszabb, még egy kicsit hosszabb, még egy kicsit hosszabb, még egy kicsit hossz

felsorolás
\begin{itemize}
    \item minden függvény-blokk atomikus, vagyis mindig lefut az elejétől a végéig, és nem szakad meg soha
    \item a függvény-blokkban nincsenek elágazások
    \item minden függvény-blokknak van egy maximum nagysága
\end{itemize}


Monospace betűtípus \texttt{fetch}, \texttt{decode} és \texttt{execute}

\paragraph{Nevesített paragrafus.} A {\it dőlt}  idézet~\cite{Ubershaders:ARidiculous}.

Beilleszteni egy képet így kell: A képre mindig kell hivatkozni a szövegből!!!

\begin{figure}[ht]
    \centering
    \includesvg[scale=0.13,inkscapelatex=false]{./kep/utasitas-abra.svg}
	\caption{Utasítás felépítése}
	\label{fig:Utasítás felépítése}
\end{figure}

\newcommand{\CS}{C${}^{\#}$}

\subsection{Így szedjük helyesen a \CS\ nyelvet}

% \chapterimage{kep/header2.png} % Chapter heading image
\chapter{Gyakorlati rész}

Még egy kicsit hosszabb, még egy kicsit hosszabb, még egy kicsit hosszabb, még egy kicsit hosszabb, még egy kicsit hosszabb, még egy kicsit hosszabb, még egy kicsit hosszabb, még egy kicsit hosszabb, még egy kicsit hosszabb, még egy kicsit hosszabb, még egy kicsit hosszabb, még egy kicsit hosszabb, még egy kicsit hosszabb, még egy kicsit hosszabb, még egy kicsit hosszabb, még egy kicsit hosszabb, még egy kicsit hosszabb, még egy kicsit hosszabb, még egy kicsit hosszabb, még egy kicsit hosszabb, még egy kicsit hosszabb, még egy kicsit hosszabb, még egy kicsit hosszabb, még egy kicsit hosszabb, még egy kicsit hosszabb, még egy kicsit hosszabb, még egy kicsit hosszabb\footnote{lábjegyzet}, még egy kicsit hosszabb, még egy kicsit hosszabb, még egy kicsit hosszabb, még egy kicsit hosszabb, még egy kicsit hosszabb, még egy kicsit hosszabb, még egy kicsit hosszabb, még egy kicsit hosszabb, még egy kicsit hosszabb, még egy kicsit hosszabb, még egy kicsit hosszabb, még egy kicsit hosszabb, még egy kicsit hosszabb, még egy kicsit hosszabb, még egy kicsit hosszabb, még egy kicsit hossz

% \chapterimage{kep/header.png} % Chapter heading image
\chapter*{Befejezés}
\addcontentsline{toc}{chapter}{Befejezés}

Még egy kicsit hosszabb, még egy kicsit hosszabb, még egy kicsit hosszabb, még egy kicsit hosszabb, még egy kicsit hosszabb, még egy kicsit hosszabb, még egy kicsit hosszabb, még egy kicsit hosszabb, még egy kicsit hosszabb, még egy kicsit hosszabb, még egy kicsit hosszabb, még egy kicsit hosszabb, még egy kicsit hosszabb, még egy kicsit hosszabb, még egy kicsit hosszabb, még egy kicsit hosszabb, még egy kicsit hosszabb, még egy kicsit hosszabb, még egy kicsit hosszabb, még egy kicsit hosszabb, még egy kicsit hosszabb, még egy kicsit hosszabb, még egy kicsit hosszabb, még egy kicsit hosszabb, még egy kicsit hosszabb, még egy kicsit hosszabb, még egy kicsit hosszabb\footnote{lábjegyzet}, még egy kicsit hosszabb, még egy kicsit hosszabb, még egy kicsit hosszabb, még egy kicsit hosszabb, még egy kicsit hosszabb, még egy kicsit hosszabb, még egy kicsit hosszabb, még egy kicsit hosszabb, még egy kicsit hosszabb, még egy kicsit hosszabb, még egy kicsit hosszabb, még egy kicsit hosszabb, még egy kicsit hosszabb, még egy kicsit hosszabb, még egy kicsit hosszabb, még egy kicsit hossz


\pagebreak
% \chapterimage{kep/header3.png} % Chapter heading image
\chapter*{Resumé}
\addcontentsline{toc}{chapter}{Resumé}
\markboth{}{\sffamily\normalsize{Resumé}}

Autor vyrešil úlohu, ešte trošku dlhšie, ešte trošku dlhšie, ešte trošku dlhšie, ešte trošku dlhšie, ešte trošku dlhšie, ešte trošku dlhšie, ešte trošku dlhšie, ešte trošku dlhšie, ešte trošku dlhšie, ešte trošku dlhšie, ešte trošku dlhšie, ešte trošku dlhšie, ešte trošku dlhšie, ešte trošku dlhšie, ešte trošku dlhšie, ešte trošku dlhšie, ešte trošku dlhšie, ešte trošku dlhšie, ešte trošku dlhšie, ešte trošku dlhšie, ešte trošku dlhšie, ešte trošku dlhšie, ešte trošku dlhšie, ešte trošku dlhšie, ešte trošku dlhšie, ešte trošku dlhšie, ešte trošku dlhšie, ešte trošku dlhšie, ešte trošku dlhšie, ešte trošku dlhšie, ešte trošku dlhšie, ešte trošku dlhšie, ešte trošku dlhšie, ešte trošku dlhšie, ešte trošku dlhšie, ešte trošku dlhšie, ešte trošku dlhšie, ešte trošku dlhšie, ešte trošku dlhšie, ešte trošku dlhšie, ešte trošku dlhšie, ešte trošku dlhšie, ešte trošku dlhšie, ešte trošku dlhšie

% \input{gitlog.tex}

\printbibliography[title=References] % Output book bibliography entries
\addcontentsline{toc}{chapter}{References}
\pagebreak
\thispagestyle{empty}
\mbox{}
\vfill
\begin{Center}
\mbox{\vskip1cm}\EANisbn
\end{Center}

\end{document}
